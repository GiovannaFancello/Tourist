\RequirePackage[l2tabu, orthodox]{nag}
\documentclass[version=last, pagesize, twoside=semi, DIV=calc, 12pt, a4paper, french, english]{scrartcl}
%INSTALL

%avoids a warning
\usepackage[log-declarations=false]{xparse}
\usepackage{fontspec} %font selecting commands
\usepackage{xunicode}
\usepackage{ucharclasses}
%warn about missing characters
\tracinglostchars=2

%REDAC
\usepackage{booktabs}
\usepackage{calc}

\usepackage{mathtools} %load this before babel!
	\mathtoolsset{showonlyrefs,showmanualtags}

\usepackage[french, english]{babel}
%suppresses the warning about frenchb not modifying the captions (“—” to “:” in “Figure 1 – Legend”).
	\frenchbsetup{AutoSpacePunctuation=false,SuppressWarning=true}

%\usepackage[super]{nth}%better use fmtcount! (loaded by datetime anyway; see below about pbl with warnings and package silence)
\usepackage{listings} %typeset source code listings
	\lstset{language=XML,tabsize=2,captionpos=b,basicstyle=\NoAutoSpacing}%NoAutoSpacing avoids space before colon or ?}%,literate={"}{{\tt"}}1, keywordstyle=\fontspec{Latin Modern Mono Light}\textbf, emph={String, PreparedStatement}, emphstyle=\fontspec{Latin Modern Mono Light}\textbf, language=Java, basicstyle=\small\NoAutoSpacing\ttfamily, aboveskip=0pt, belowskip=0pt, showstringspaces=false
\usepackage[nolist,smaller,printonlyused]{acronym}%,smaller option produces warnings from relsize in some cases, it seems.% Note silence and acronym and hyperref make (xe)latex crash when ac used in section (http://tex.stackexchange.com/questions/103483/strange-packages-interaction-acronyms-silence-hyperref), rather use \section{\texorpdfstring{\acs{UE}}{UE}}.
\usepackage{fmtcount}
\usepackage[nodayofweek]{datetime}%must be loaded after the babel package. However, loading it after {nth} generates a warning from fmtcount about ordinal being already defined. Better load it before nth? (then we can remove the silence package which creates possible crashes, see above.) Or remove nth?
%\usepackage{xspace}%do we need this?
\usepackage[textsize=small]{todonotes}
\usepackage[pdfusetitle]{hyperref}% option pdfusetitle must be introduced here, not in hypersetup.
%breaklinks makes links on multiple lines into different PDF links to the same target.
%colorlinks (false): Colors the text of links and anchors. The colors chosen depend on the the type of link. In spite of colored boxes, the colored text remains when printing.
%linkcolor=black: this leaves other links in colors, e.g. refs in green, don't print well.
%pdfborder (0 0 1, set to 0 0 0 if colorlinks): width of PDF link border
%hidelinks
\hypersetup{breaklinks, bookmarksopen}
%in Beamer, sets url colored links but does not change the rest of the colors (http://tex.stackexchange.com/questions/13423/how-to-change-the-color-of-href-links-for-real)
%\hypersetup{breaklinks,bookmarksopen,colorlinks=true,urlcolor=blue,linkcolor=,hyperfigures=true}
% hyperref doc says: Package bookmark replaces hyperref’s bookmark organization by a new algorithm (...) Therefore I recommend using this package.
\usepackage{bookmark}

% center floats by default, but do not use with float
% \usepackage{floatrow}
% \makeatletter
% \g@addto@macro\@floatboxreset\centering
% \makeatother
\usepackage{enumitem} %follow enumerate by a string saying how to display enumeration
\usepackage{ragged2e} %new com­mands \Cen­ter­ing, \RaggedLeft, and \RaggedRight and new en­vi­ron­ments Cen­ter, FlushLeft, and FlushRight, which set ragged text and are eas­ily con­fig­urable to al­low hy­phen­ation (the cor­re­spond­ing com­mands in LaTeX, all of whose names are lower-case, pre­vent hy­phen­ation al­to­gether). 
\usepackage{siunitx} %[expproduct=tighttimes, decimalsymbol=comma] ou (plus récent ?) [round-mode=figures, round-precision=2, scientific-notation = engineering]
\sisetup{detect-all, locale = FR, strict}% to detect e.g. when in math mode (use a math font) - check whether this makes sense with strict
\usepackage{braket} %for \Set
\usepackage{natbib}
\usepackage{doi}

\usepackage{amsmath,amsthm}
% \usepackage{amsfonts} %not required?
% \usepackage{dsfont} %for what?
%unicode-math overwrites the following commands from the mathtools package: \dblcolon, \coloneqq, \Coloneqq, \eqqcolon. Using the other colon-like commands from mathtools will lead to inconsistencies. Plus, Using \overbracket and \underbracke from mathtools package. Use \Uoverbracket and \Uunderbracke for original unicode-math definition.
%use exclusively \mathbf and choose math bold style below.
\usepackage[warnings-off={mathtools-colon, mathtools-overbracket}, bold-style=ISO]{unicode-math}

\defaultfontfeatures{
	Fractions=On,
	Mapping=tex-text% to turn "--" into dashes, useful for bibtex%%
}
%\defaultfontfeatures[\rmfamily]{
%	Fractions=On,
%	Mapping=% to leave " alone (disable the default mapping tex-text; requires loading the font afterwards?)
%}
\newfontfamily\xitsfamily{XITS}
\newfontfamily\texgyretermesfamily{TeX Gyre Termes}
\newfontfamily\lmfamily{Latin Modern Roman}
\setmainfont{TeX Gyre Termes}
\setsansfont{TeX Gyre Heros}
\setmonofont{TeX Gyre Cursor}
\defaultfontfeatures{}%disable default font features to avoid warnings with math fonts.
\setmathfont{XITS Math}
\setmathfont[range={\mathcal,\mathbfcal},StylisticSet=1]{XITS Math}
%silently delays restoring normal font when using e.g. “≠)” or “≠ other word”.
%\setTransitionTo{Arrows}{\xitsfamily}
%\setTransitionTo{MathematicalOperators}{\xitsfamily}
%this solution explicitly fails on “≠)” and works for “≠ other word.”
\setTransitionsFor{MathematicalOperators}{\begingroup\xitsfamily}{\endgroup}
\setTransitionsFor{Arrows}{\begingroup\xitsfamily}{\endgroup}
%texgyretermes does not have ✓, XITS does.
\setTransitionsFor{Dingbats}{\begingroup\xitsfamily}{\endgroup}
%this also works, but it is more complex
%\def\ResetTransitionTo#1{%
%  \XeTeXinterchartoks 255 \csname#1Class\endcsname{\relax}}
%\setTransitionsFor{MathematicalOperators}
%  {\begingroup\ResetTransitionTo{MathematicalOperators}\xitsfamily}
%  {\endgroup}

\usepackage{cleveref}% cleveref should go "laster" than hyperref
%GRAPHICS
\usepackage{pgf}
\usepackage{pgfplots}
	\usetikzlibrary{babel, matrix, fit, plotmarks, calc, trees, shapes.geometric, positioning, plothandlers, arrows, shapes.multipart}
\pgfplotsset{compat=1.11}
\usepackage{graphicx}

\graphicspath{{graphics/},{graphics-dm/}}
\DeclareGraphicsExtensions{.pdf}
\LetLtxMacro\SavedIncludeGraphics\includegraphics
\AtBeginDocument{
	\def\includegraphics#1#{% #1 catches optional stuff (star/opt. arg.)
		\IncludeGraphicsAux{#1}%
	}%
}
\newcommand*{\IncludeGraphicsAux}[2]{%
	\XeTeXLinkBox{%
		\SavedIncludeGraphics#1{#2}%
	}%
}%

%HACKING
\usepackage{printlen}
\uselengthunit{mm}
% 	\newlength{\templ}% or LenTemp?
% 	\setlength{\templ}{6 pt}
% 	\printlength{\templ}
\usepackage{etoolbox} %for addtocmd command
\usepackage{scrhack}% load at end. Corrects a bug in float package, which is outdated but might be used by other packages
\usepackage{xltxtra} %somebody said that this is loaded by fontspec, but does not seem correct: if not loaded explicitly, does not appear in the log and \showhyphens is not corrected.


\newcommand{\R}{ℝ}
\newcommand{\N}{ℕ}
\newcommand{\Z}{ℤ}
\newcommand{\card}[1]{\lvert{#1}\rvert}
\newcommand{\powerset}[1]{\mathscr{P}(#1)}%\mathscr rather than \mathcal: scr is rounder than cal (at least in XITS Math).
\newcommand{\suchthat}{\;\ifnum\currentgrouptype=16 \middle\fi|\;}
%\newcommand{\Rplus}{\reels^+\xspace}

\AtBeginDocument{%
	\renewcommand{\epsilon}{\varepsilon}
% we want straight form of \phi for mathematics, as recommended in UTR #25: Unicode support for mathematics.
%	\renewcommand{\phi}{\varphi}
}

% with amssymb, but I don’t want to use amssymb just for that.
% \newcommand{\restr}[2]{{#1}_{\restriction #2}}
%\newcommand{\restr}[2]{{#1\upharpoonright}_{#2}}
\newcommand{\restr}[2]{{#1|}_{#2}}%sometimes typed out incorrectly within \set.
%\newcommand{\restr}[2]{{#1}_{\vert #2}}%\vert errors when used within \Set and is typed out incorrectly within \set.
\DeclareMathOperator*{\argmax}{arg\,max}
\DeclareMathOperator*{\argmin}{arg\,min}


%ARG TH
\newcommand{\AF}{\mathcal{AF}}
\newcommand{\labelling}{\mathcal{L}}
\newcommand{\labin}{\textbf{in}\xspace}
\newcommand{\labout}{\textbf{out}}
\newcommand{\labund}{\textbf{undec}\xspace}
\newcommand{\nonemptyor}[2]{\ifthenelse{\equal{#1}{}}{#2}{#1}}
\newcommand{\gextlab}[2][]{
	\labelling{\mathcal{GE}}_{(#2, \nonemptyor{#1}{\ibeatsr{#2}})}
}
\newcommand{\allargs}{A^*}
\newcommand{\args}{A}
\newcommand{\ar}{a}
\newcommand{\ext}{\mathcal{E}}

%MCDA+Arg
\newcommand{\dm}{d}
\newcommand{\ileadsto}{\rightcurvedarrow}
\newcommand{\mleadsto}[1][\eta]{\rightcurvedarrow_{#1}}
\newcommand{\ibeats}{\vartriangleright}
\newcommand{\mbeats}[1][\eta]{\vartriangleright_{#1}}


%MISC
\newcommand{\lequiv}{\Vvdash}
\newcommand{\weightst}{W^{\,t}}

%MCDA classical
\newcommand{\crits}{\mathcal{J}}
\newcommand{\allalts}{\mathcal{A}}
\newcommand{\alts}{A}

%Sorting
\newcommand{\cats}{\mathcal{C}}
\newcommand{\catssubsets}{2^\cats}
\newcommand{\catgg}{\vartriangleright}
\newcommand{\catll}{\vartriangleleft}
\newcommand{\catleq}{\trianglelefteq}
\newcommand{\catgeq}{\trianglerighteq}
\newcommand{\alttoc}[2][x]{(#1 \xrightarrow{} #2)}
\newcommand{\alttocat}[3]{(#2 \xrightarrow{#1} #3)}
\newcommand{\alttoI}{(x \xrightarrow{} \left[\underline{C_x}, \overline{C_x}\right])}
\newcommand{\alttocatdm}[3][t]{\left(#2 \thinspace \raisebox{-3pt}{$\xrightarrow{#1}$}\thinspace #3\right)}
\newcommand{\alttocatatleast}[2]{\left(#1 \thinspace \raisebox{-3pt}{$\xrightarrow[]{≥}$}\thinspace #2\right)}
\newcommand{\alttocatatmost}[2]{\left(#1 \thinspace \raisebox{-3pt}{$\xrightarrow[]{≤}$}\thinspace #2\right)}

\newcommand{\commentOC}[1]{{\selectlanguage{french}{\todo{OC : #1}}}}
%Or: \todo[color=green!40]
\newcommand{\innote}[1]{{\scriptsize{#1}}}

%this probably requires outdated float package, see doc KomaScript for an alternative.
% \newfloat{program}{t}{lop}
% \floatname{program}{PM}

%style is plain by default (italic text)
	\newtheorem{definition}{Definition}
	\newtheorem{theorem}{Theorem}
%no italic: expected.
%http://tex.stackexchange.com/questions/144653/italicizing-of-amsthm-package
	\newtheorem{lemma}{Lemma}
%\crefname{axiom}{axiom}{axioms}%might be needed for workaround bug in cref when defining new theorems?

%\ifdefined\theorem\else
%\newtheorem{theorem}{\iflanguage{english}{Theorem}{Théorème}}
%\fi

\theoremstyle{remark}
	\newtheorem{examplex}{Example}
	\newtheorem{remarkx}{Remark}

%trickery allowing use of \qedhere and the like.
\newenvironment{example}{
	\pushQED{\qed}\renewcommand{\qedsymbol}{$\triangle$}\examplex
}{
	\popQED\endexamplex
}
\newenvironment{remark}{
	\pushQED{\qed}\renewcommand{\qedsymbol}{$\triangle$}\remarkx
}{
	\popQED\endremarkx
}

%which line breaks are chosen: accept worse lines, therefore reducing risk of overfull lines. Default = 200
\tolerance=2000
%accept overfull hbox up to...
\hfuzz=2cm
%reduces verbosity about the bad line breaks
\hbadness 5000
%sloppy sets tolerance to 9999
\apptocmd{\sloppy}{\hbadness 10000\relax}{}{}

\bibliographystyle{abbrvnat}
%or \bibliographystyle{apalike} for presentations?

%doi package uses old-style dx.doi url, see 3.8 DOI system Proxy Server technical details, “Users may resolve DOI names that are structured to use the DOI system Proxy Server (http://doi.org (preferred) or http://dx.doi.org).”, https://www.doi.org/doi_handbook/3_Resolution.html
\makeatletter
\patchcmd{\@doi}{dx.doi.org}{doi.org}{}{}
\makeatother

% WRITING
%\newcommand{\ie}{i.e.\@\xspace}%to try
%\newcommand{\eg}{e.g.\@\xspace}
%\newcommand{\etal}{et al.\@\xspace}
\newcommand{\ie}{i.e.\ }
\newcommand{\eg}{e.g.\ }
\newcommand{\mkkOK}{\checkmark}%\color{green}{\checkmark}
\newcommand{\mkkREQ}{\ding{53}}%requires pifont?%\color{green}{\checkmark}
\newcommand{\mkkNO}{}%\text{\color{red}{\textsf{X}}}

\makeatletter
\newcommand{\boldor}[2]{%
	\ifnum\strcmp{\f@series}{bx}=\z@
		#1%
	\else
		#2%
	\fi
}
\newcommand{\textstyleElProm}[1]{\boldor{\MakeUppercase{#1}}{\textsc{#1}}}
\makeatother
\newcommand{\electre}{\textstyleElProm{Électre}\xspace}
\newcommand{\electreIv}{\textstyleElProm{Électre Iv}\xspace}
\newcommand{\electreIV}{\textstyleElProm{Électre IV}\xspace}
\newcommand{\electreIII}{\textstyleElProm{Électre III}\xspace}
\newcommand{\electreTRI}{\textstyleElProm{Électre Tri}\xspace}
% \newcommand{\utadis}{\texorpdfstring{\textstyleElProm{utadis}\xspace}{UTADIS}}
% \newcommand{\utadisI}{\texorpdfstring{\textstyleElProm{utadis i}\xspace}{UTADIS I}}

%TODO
% \newcommand{\textstyleElProm}[1]{{\rmfamily\textsc{#1}}} 


\newlength{\GraphsNodeSep}
\setlength{\GraphsNodeSep}{7mm}

% MCDA Drawing Sorting
\newlength{\MCDSCatHeight}
\setlength{\MCDSCatHeight}{6mm}
\newlength{\MCDSAltHeight}
\setlength{\MCDSAltHeight}{4mm}
%separation between two vertical alts
\newlength{\MCDSAltSep}
\setlength{\MCDSAltSep}{2mm}
\newlength{\MCDSCatWidth}
\setlength{\MCDSCatWidth}{3cm}
\newlength{\MCDSEvalRowHeight}
\setlength{\MCDSEvalRowHeight}{6mm}
\newlength{\MCDSAltsToCatsSep}
\setlength{\MCDSAltsToCatsSep}{1.5cm}
\newcounter{MCDSNbAlts}
\newcounter{MCDSNbCats}
\newlength{\MCDSArrowDownOffset}
\setlength{\MCDSArrowDownOffset}{0mm}

\tikzset{/Graphs/dot/.style={
	shape=circle, fill=black, inner sep=0, minimum size=1mm
}}
\tikzset{/MC/D/S/alt/.style={
	shape=rectangle, draw=black, inner sep=0, minimum height=\MCDSAltHeight, minimum width=2.5cm, anchor=north east
}}
\tikzset{MC/D/S/pref/.style={
	shape=ellipse, draw=gray, thick
}}
\tikzset{/MC/D/S/cat/.style={
	shape=rectangle, draw=black, inner sep=0, minimum height=\MCDSCatHeight, minimum width=\MCDSCatWidth, anchor=north west
}}
\tikzset{/MC/D/S/evals matrix/.style={
	matrix, row sep=-\pgflinewidth, column sep=-\pgflinewidth, nodes={shape=rectangle, draw=black, inner sep=0mm, text depth=0.5ex, text height=1em, minimum height=\MCDSEvalRowHeight, minimum width=12mm}, nodes in empty cells, matrix of nodes, inner sep=0mm, outer sep=0mm, row 1/.style={nodes={draw=none, minimum height=0em, text height=, inner ysep=1mm}}
}}

% GUI
\tikzset{/GUI/button/.style={
	rectangle, very thick, rounded corners, draw=black, fill=black!40%, top color=black!70, bottom color=white
}}

% Beliefs
\tikzset{/Beliefs/D/S/attacker/.style={
	shape=rectangle, draw, minimum size=8mm
}}
\tikzset{/Beliefs/D/S/supporter/.style={
	shape=circle, draw
}}

\newcommand{\tikzmark}[1]{%
	\tikz[overlay, remember picture, baseline=(#1.base)] \node (#1) {};%
}


\begin{acronym}
\acro{AMCD}{Aide Multicritère à la Décision}
\acro{ASA}{Argument Strength Assessment}
\acro{DA}{Decision Analysis}
\acro{DM}{Decision Maker}
\acro{DPr}{Deliberated Preferences}
\acro{DRSA}{Dominance-based Rough Set Approach}
\acro{DSS}{Decision Support Systems}
\acrodefplural{DSS}{Decision Support Systems}
% \newacroplural{DSS}[DSSes]{Decision Support Systems}
\acro{EJOR}{European Journal of Operational Research}
\acro{LNCS}{Lecture Notes in Computer Science}
\acro{MCDA}{Multicriteria Decision Aid}
\acro{MIP}{Mixed Integer Program}
\acro{NCSM}{Non Compensatory Sorting Model}
\acro{PL}{Programme Linéaire}
\acro{PLNE}{Programme Linéaire en Nombres Entiers}
\acro{PM}{Programme Mathématique}
\acro{MP}{Mathematical Program}
\acro{MIP}{Mixed Integer Program}
% \newacroplural{PM}{Programmes Mathématiques}
%acrodefplural since version 1.35, my debian has \ProvidesPackage{acronym}[2009/01/25, v1.34, Support for acronyms (Tobias Oetiker)]
\acrodefplural{PM}{Programmes Mathématiques}
\acro{PMML}{Predictive Model Markup Language}
\acro{RESS}{Reliability Engineering \& System Safety}
\acro{SMAA}{Stochastic Multicriteria Acceptability Analysis}
\acro{URPDM}{Uncertainty and Robustness in Planning and Decision Making}
\acro{XML}{Extensible Markup Language}
\end{acronym}


\begin{document}
\title{Tourists'preferences ...}
\author{Olivier Cailloux, Giovanna Fancello, Alexis Tsoukiàs,...}
\makeatletter
	\hypersetup{
		pdfsubject={MAVT},
		pdfkeywords={MAVT, preference model, MCDA}
	}
\makeatother
\maketitle

\tableofcontents

\section{Introduction}
\begin{itemize}
   \item [-]state of the art and aim of the paper
   \item [-]the tourist as one of the population in the city
   \item [-]the possibility to access to some urban opportunity in order to improve their urban-quality of life (in a capability theory point of view)
 \end{itemize}
\textbf{Main Objective.}
To define the tourists values in space in respect to the correspondent tourist socio-professional class and behaviours.

\section{Alghero case study}

\textbf{Case Study.}
The city of Alghero and his territory in the 2014 touristic "low season" (October-November) considering that Alghero's peek period is the summer season, with highest tourist concentration between July and September. Data collection is carried out in a touristic low season period because we want to catch urban limits and opportunities in a period of reduction of urban activities and not bathing season and in order to have significant results for a tourist public policy aimed to deseasonalize this trend.
\textbf{Data}
We have 75 questionnaires representing \textbf{225 tourists $T$} described by a \textbf{set of attributes $A$}
\begin{equation}
A=\{gender,age,country,level of study,profession,willingness to pay\}
\end{equation}
We know \textbf{tourists' paths} in the territory, in the space described by \textbf{Coordinates} and \textbf{Time}.
Let $S$ denote the set of possible coordinates and $\tau$ the set of possible times.  A path is a set of points $P \subseteq S \times \tau$.\\
Finally we define a set of \textbf{Categories of places} $C \in S$ that a tourist can choose in Alghero city.
Let $C=\{c_1, \ldots, c_7\}$ denote the set of categories of places.

\begin{table}[h]
\begin{tabular}{ll}
\hline
\textbf{Categories of places}&\textbf{Places}\\
\hline
\parbox{5cm}{Environmental elements (local)}&\parbox{6cm}{Lido, M. Pia, \dots}\\
\parbox{5cm}{Environmental elements (territorial)}&\parbox{6cm}{Grotte di Nettuno, Punta Giglio, Spiaggia del Lazzaretto, \dots}\\
\parbox{5cm}{Historical and archaeological elements (local)}&\parbox{6cm}{Cattedrale,Bastioni,Historical centre, \dots}\\
\parbox{5cm}{Historical and archaeological elements (territorial)}&\parbox{6cm}{Fertilia, Castelsardo, Stintino, nuraghe Palmavera, \dots}\\
\parbox{5cm}{Cultural Elements}&\parbox{6cm}{Theater, Cinema, Museum, \dots}\\
\parbox{5cm}{Food services}&\parbox{6cm}{Restaurants, Market,\dots}\\
\parbox{5cm}{Leisure}&\parbox{6cm}{Waterfront, Public Gardens, Harbor, \dots}\\
\parbox{5cm}{Other}&\parbox{6cm}{Stay in the Hotel, friends' home, Route from one place to another, \dots}\\
\hline
\end{tabular}

\caption{\textbf{Categories of places}}\
\label{Table1}
\end{table}

Starting with these considerations, we define a \textbf{vector path} $x: C \rightarrow R^+$ as a function that maps each category of places to a number of seconds.
We associate to each tourist $t \in T $ a vector path $x^t$.
Let $X={\mathbb{R}^+}^C$ denote the set of all possible vector paths.

We define a preference relation $S \subseteq X \times X $ as a binary relation over the set of possible vector paths.

Our objective is to associate to each tourist $t \in T$ a preference relation $S_t$ which represents according to our model the way the tourist evaluates the possible paths in the Alghero territory. A preference relation is given when the tourist can freely choose a path to another in respect to his individual characteristics (age, gender,\dots) and to his personal (income,\dots) and spatial resources (means of transport, \dots) (SEN). Let $X^t {\subseteq}X$ denote the set of all vector paths that can be freely  chosen by $t$.


We represent preference relations using additive value functions. Let $u^t$ denote the value function of tourist $t$. We define $S_t$ from  $u^t$ as follows:
\begin{equation}
(x_1,x_2) \in S_t \text { iff } u^t(x_1) \geq u^t(x_2).
\end{equation}

Our objective is thus to obtain $u^t$ from the path data.

We assume that $u^t$ can be represented as a weighted sum  of partial value functions:
\begin{equation}
u^t(x) = \sum_{c \in C} u_c(x_c) w^t_c,
\end{equation}
where $w^t \in [0,1]$ represents the weight that the tourist $t$ gives to the category $c$.

We assume for now that the partial value functions $\{u_c\}$ are the same for each tourists $t \in T$ (\textcolor{red}{or not?}), but the weights may depend on tourist $t$.

Given a category $c \in C$, we define a partial value function $u_c:{\mathbb{R}^+} \rightarrow [0,1]$. The number $u_c(x_c)$ represents the value we assume the tourist gives to spending $x_c$ seconds in the category $c$, not taking into account the partial weight $w^t_c$.

We define each $u_c$ as a two linear pieces increasing function determined by the UTA method (\textcolor{red}{TO BE COMPLETED} \dots) .

In order to define a set of  possible tourists' preferences we consider the relation among the path chosen by the tourist and a set of outstanding paths.The outstanding paths are defined by a combination of categories of places $C$ and time $\tau$ they ideally spend in. A tourist's preference is verified if the tourist $t$ chooses a path internal to the his choice set $x_1 \in X^t$ to another $x_2 \in X^t$. We define an outstanding path as a path that a tourist can freely choose in respect to his personal characteristics and spatial and personal resources (\textcolor{red}{TO BE COMPLETED} \dots). We define three possible set of outstanding path:
                                            \begin{itemize}
                                                \item \textbf{Hard set}: all the tourist paths can be considered as outstanding paths. This means that a tourist $t$ prefers his path to all the other tourists' paths. But, as each tourist has his personal set of possible paths (in a capability framework), it is possible that $X^{t_1}\cap X^{t_2}$ or that $X^{t_1}\neq X^{t_2}$.
                                                    \subitem {-} \textcolor{red}{but not all the people has the same characteristics and the same resources, so we need to define different outstanding paths.}
                                                \item \textbf{Soft set}: a set of outstanding paths, not expensive and reachable on foot. This permits to consider a sort of basic set of paths (that every tourist can freely choose). We define five outstanding paths $x_{O_1}, \ldots, x_{O_5}$ that we assume tourists have considered. We assume that the tourist $t$ prefers the path $x^t$ he has chosen to each of the outstanding paths: $u^t(x^t) \geq u^t(x_{O_i}), 1 \leq  i \leq 5$.
                                                    \textcolor{red}{
                                                    \subitem - how we can define this set?
                                                    \subitem - We started to do this with 5 outstanding paths but results demonstrate the necessity to enlarge the set of path considered.}
                                                \item \textbf{Soft set2}: a set of outstanding paths that we are sure that every tourist can do and that he doesn't prefer: for example, to spend all the day in a category of place.We define eight outstanding paths $x_{O_1}, \ldots, x_{O_8}$ that we assume tourists have considered.We assume that the tourist $t$ prefers the path $x^t$ he has chosen to each of the outstanding paths: $u^t(x^t) \geq u^t(x_{O_i}), 1 \leq  i \leq 8$. This set of outstanding paths is used for learning $u^t$. Then we validate this results verifying if $u^t(x^t)\geq u^t(x_{O_i}), 1 \leq  i \leq 5$ using the \textbf{Soft set} of outstanding path.
                                                    \textcolor{red}{
                                                    \subitem - We started to do this but the software seems to have problems because gives for each tourist the same result and results are not correct.}
                                              \end{itemize}





\begin{table}[h]
\centering
\begin{tabular}{ll}
\hline
\textbf{Outstanding paths}\\
\hline
\textbf{Soft set}\\
\parbox{2cm}{Op1}&\parbox{8cm}{Environmental local (6h),Cultural (1h) and Food services (3h), other (4h)}\\
\parbox{2cm}{Op2}&\parbox{8cm}{Environmental territorial(7h) and Food services (3h), other (5h)}\\
\parbox{2cm}{Op3}&\parbox{8cm}{Historical local (2h), Cultural (2h), Leisure (6h) and Food services (2h), other (3h)}\\
\parbox{2cm}{Op4}&\parbox{8cm}{Historical local(4h), Leisure(3h), and Food services (1h), other (7h)}\\
\parbox{2cm}{Op5}&\parbox{8cm}{Historical territorial (7h), Historical local(3h), and Food services (3h), other (2h)}\\
\hline
\textbf{Soft set2}\\
\parbox{2cm}{Op1}&\parbox{8cm}{Environmental local (15h)}\\
\parbox{2cm}{Op2}&\parbox{8cm}{Environmental territorial(15h)}\\
\parbox{2cm}{Op3}&\parbox{8cm}{Historical local(15h)}\\
\parbox{2cm}{Op4}&\parbox{8cm}{Historical territorial (15h)}\\
\parbox{2cm}{Op5}&\parbox{8cm}{Cultural (15h)}\\
\parbox{2cm}{Op6}&\parbox{8cm}{Food services (15h)}\\
\parbox{2cm}{Op7}&\parbox{8cm}{Leisure(15h)}\\
\parbox{2cm}{Op8}&\parbox{8cm}{Other (15h)}\\
\hline
\end{tabular}
\caption{\textbf{Outstanding paths}}\
\label{Table}
\end{table}





\section{Method}


\begin{enumerate}
\item \textbf{Splitting the space} Subdivide the territory in different spaces $s$.
\item \textbf{Classification of spaces.} Classify each space $s$ in a Category of places $c$
\item \textbf{Counting the time.} For each tourists' path we analyse the time spent for each category of place $c$.For each $s$ we analyse how much time $T$ the tourists $t$ spent in it. The time spent in each category is given by the  sum of the different ranges of time in this category of place. We consider that each tourist has 15 hours to spend in a day.
\item \textbf{Value functions}. We define the value function for each tourist with the UTA method and the DIVIZ software.
\item \textbf{Defining weights}. We define the importance that each tourist gives to visit the different spaces.
\item \textbf{Clustering to define tourists' profiles}.

\end{enumerate}


\section{Results}
\dots Results

\section{Conclusions}
\dots results \dots


\section{other data}
We know the \textbf{tourist declared preferences} (why they choose to stay in Alghero and what they want to do). The criteria the tourists used to choose to visit Alghero are ordered by importance with an evaluation scale. Let's have for each tourists' criteria $zt \in Zt$ a value $n \in \xi$ with $\xi\{1,2,3,4,5\}$
\begin{equation}
\begin{split}
Zt&=n\{Economy,Environment,Weather,Food,Culture,Recreation,\\
&Entertainment,Study,Work,Relax,Friends and relatives,others\}
\end{split}
\end{equation}


We know how the tourists planned to spend their time during the holidays in Alghero.  Let's have for each tourists' action $at\in At$, a value $m \in \lambda$ with $\lambda\{1,0\}$
\begin{equation}
\begin{split}
At=m\{Environmental,Surrounding,Leisure,Fun,Food,Cultural,\\
& Work\}
\end{split}
\end{equation}

\appendix
\section{A simple goal}
Data:
\begin{labeling}{$d_t: Z → \set{1, 2, 3, 4, 5}$}
	\item[$T$] Set of tourists ($t$ a tourist)
	\item[$Z$] Set of aspects on which tourist has declared a degree of interest (here above, $Zt$ or $At$)
	\item[$d^t: Z → \set{1, 2, 3, 4, 5}$] A function such that $d^t(z) \in \set{1, 2, 3, 4, 5}$ indicates, for tourist $t \in T$ and aspect $z \in Z$, the degree of interest on aspect $z$ as declared by $t$
	\item[$i^t \in \R^+$] Intensity corresponding to $t$: the total number of seconds tourist $t$ has spent on her trip (from leaving home to returning home)
	\item[$C_V$] The categories of places in Alghero \emph{that can be visited}, thus \emph{excluding} staying home
	\item[$x^t: C_V → \R^+$] The vector path chosen by tourist $t$ (associates a number of seconds to each category of place). By definition, $\sum_{c \in C_V} x^t(c) = i^t$.
\end{labeling}

Our goal is to predict $x^t$ given $d^t$ and an intensity: a number of seconds that indicates how long the tourist intends to visit for. We are allowed to use $T_1 \subseteq T$ as training data, and must use $T_2 \subseteq T$ as test data. Given a predictor $P: {\set{1, 2, 3, 4, 5}}^Z × \R^+ → {(\R^+)}^{C_V}$, the quality of $P$ is the sum, on the test data, of the distance, using L2 norm, between the prediction and the real path: quality of $P$ = $\sum_{t \in T_2} \|P(d^t, i^t) − x^t\|$.

Here below we list a few possible ways of building predictors, starting from the simplest ones.

\subsection{Central prediction}
We compute the centre (using L2) of $(x^t)_{t \in T_1}$ and constantly predict this.

\subsection{Regression}
Build some regression model between ${\set{1, 2, 3, 4, 5}}^Z × \R^+$ and ${(\R^+)}^{C_V}$.

\subsection{Guided regression}
Same idea but we (somehow) use knowledge about the link between $Z$ and $C_V$.

\subsection{Transformed space}
We fix (a priori) a set of partial value functions $u_c: \R^+ → [0, 1]$, one for each category of place. The vector $(u_c \circ x^t): C_V → [0, 1]$ represents the values associated by tourist $t$, given its path, to each category of places. We learn a regression model (using normal or guided regression) between ${\set{1, 2, 3, 4, 5}}^Z × \R^+$ and latent weights $w^t$, in order to maximize $w^t \cdot (u_c \circ x^t)$, the value of $x^t$.

\subsection{Learned transform}
Same idea as Transformed space, but we learn the partial value functions using $T_1$.

\subsection{More ideas}
See article Siskos. See utilitaristic regression from Eyke.
\end{document}
